\documentclass{SPANISH_acm_proc_article-sp}

\begin{document}

\title{Visitando la Plataforma Continental Argentina}
\subtitle{}

\numberofauthors{5}

\author
{
	%1st author
	\alignauthor
    Gustavo Maldonado \\
	\affaddr{Instituto Tecnol\'ogico de Buenos Aires (ITBA)} \\
	\affaddr{Buenos Aires, Argentina}\\
	\email{gmaldona@alu.itba.edu.ar}
	%2nd author
	\alignauthor
	Guido Marucci Blas \\
    \affaddr{Instituto Tecnol\'ogico de Buenos Aires (ITBA)} \\
	\affaddr{Buenos Aires, Argentina}\\
	\email{gmarucci@alu.itba.edu.ar}
	%3rd author
	\alignauthor	
    Santiago Perez De Rosso \\
    \affaddr{Instituto Tecnol\'ogico de Buenos Aires (ITBA)} \\
	\affaddr{Buenos Aires, Argentina}\\
	\email{sperezde@alu.itba.edu.ar}
	\and
	%4th author
	\alignauthor
	Nicolas Purita \\
    \affaddr{Instituto Tecnol\'ogico de Buenos Aires (ITBA)} \\
	\affaddr{Buenos Aires, Argentina}\\
	\email{npurita@alu.itba.edu.ar}	
	\and
    %5th author
    \alignauthor
    Luciano Zemin \\
    \affaddr{Instituto Tecnol\'ogico de Buenos Aires (ITBA)} \\
	\affaddr{Buenos Aires, Argentina}\\
	\email{lzemin@alu.itba.edu.ar}
}

\maketitle

\section*{RESUMEN}
En este art\'iculo se analiza el per\'iodo de evoluci\'on y su funci\'on densidad de 
probabilidad de un \emph{modelo ecol\'ogico presa - predador} a partir de una muestra
inicial y de las caracter\'isticas evolutivas del sistema (sus \emph{par\'ametros bi\'oticos}).
En particular, se analiza el caso de la zona sobre la Plataforma Continental Argentina
donde la especie \textit{Tibur\'on Pintarrojo} preda a la especie \textit{Salm\'on de Mar},
siendo esta \'ultima un importante recurso econ\'omico pesquero nacional.

\section*{Palabras Clave}
Modelo de \textit{Lokta-Volterra-Ancona}, \textit{Montecarlo}, \textit{Salm\'on de Mar},
\textit{Tibur\'on Pintarrojo}

\section{INTRODUCCI\'ON}
Es de una gran importancia econ\'omica aparte de biol\'ogica el estudio de la din\'amica
presa-predador de ciertas especies de recurso econ\'omico pesquero. En el presente art\'iculo
se analiza en particular el caso de la zona sobre la Plataforma Continental Argentina donde
la especie \emph{Salm\'on de Mar (Pseudopercis semifasciata)} es capturada desde una
latitud correspondiente a la desembocadura del R\'io Colorado, hasta el Golfo de San Juli\'an.
El \emph{Tibur\'on Pintarrojo (Haleaeulurus bivius)} es una de las cincuenta especies que
habitan la Plataforma Continental Argentina. Dicha especia preda al Salm\'on de Mar.
Ambas poblaciones est\'an en equilibrio din\'amico, pero no se tiene la suficiente 
cantidad de datos para estimar el ciclo poblacional del recurso, en este caso, el salm\'on.
El conocimiento de dicho ciclo, el cual constituye el per\'iodo de variaci\'on poblacional,
es un par\'ametro de fundamental importancia econ\'omica adem\'as de biol\'ogica. Ya que
este permite estimar los periodos de vedas o disminuci\'on de las tazas de captura
econ\'omica.

En la secci\'on \ref{sec:modelo} se presenta el modelo utilizado a lo largo del art\'iculo,
en la secci\'on \ref{sec:estimacion} se detalla el proceso de obtenci\'on de la
estimaci\'on de la funci\'on densidad de probabilidad del \emph{Per\'iodo}. Finalmente,
en la secci\'on \ref{sec:resultados} se presentan los resultados obtenidos y las
conclusiones finales.

\section{Modelizaci\'on del sistema}
\label{sec:modelo}
El sistema se modela utilizando el modelo de \textit{Lokta-Volterra-Ancona}. Se considera que el predador
se alimenta exclusivamente de la presa, mientras que esta \'ultima se alimenta de recursos ilimitados
que se encuentran en su h\'abitat. El ambiente no influye sobre el sistema, asi como el sexo o
el estado de salud de los individuos.\\
En las ecuaciones \ref{eq:modelo1} y \ref{eq:modelo2} se puede ver el modelo resultante:
\begin{equation}
\label{eq:modelo1}
\dot{x} = \lambda x - axy
\end{equation}
\begin{equation}
\label{eq:modelo2}
\dot{y} = bxy - \mu y
\end{equation}
donde $x(y)$ e $y(t)$ representan las poblaciones de la especie presa y predador, 
respectivamente. Las constantes $\lambda$, $\mu$ representan las tasa de crecimiento
poblacional de presas y predadores respectivamente, en ausencia de sus contrapartes.
Las constantes positivas $a$ y $b$ representan las tasas de encuentros perjudiciales para 
las presas y beneficiosos para predadores.\\


\section{Estimaci\'on de la funci\'on densidad de probabilidad del \emph{Per\'iodo}}
\label{sec:estimacion}
Con el objetivo de estimar la funci\'on densidad de probablidad del \emph{Per\'iodo}
se utiliza el m\'etodo de \textit{Monte Carlo}. El m\'etodo de \textit{Monte Carlo} es una metodolog\'ia
para obtener, mediante simulaci\'on, una muestra estad\'istica y computar estimadores
de par\'ametros para poblaciones mayores \textit{(Diaz, 2011)}. Las simulaciones fueron
realizadas con \textit{Matlab}. Para la resoluci\'on de las ecuaciones diferenciales se
 utiliza \textit{ODE45} que es el recomendado en
\textit{Matlab},est\'a basado en la f\'ormula expl\'icita de Runge-Kutta (4, 5) Dormand-Prince. Utiliza seis
funciones de valuaci\'on para calcular soluciones de precisi\'on de 4to y 5to orden. Este m\'etodo se considera
adaptativo ya que var\'ia el tama\~no de paso de acuerdo al error \textit{(Mathworks, 2011)}.

Asumiendo un valor estimado de $ \hat b = 0.0035 \, $a\~nos$^{-1} $ y suponiendo que
$ a = (0.02 \, \underline{+} \, 0.001) \, $a\~nos$^{-1}$ uniformemente distribuida,
$ \lambda = N(1.5 \, $a\~nos$^{-1}, 0.10) $ es una funci\'on de distribuci\'on de probabilidad normal de media
$ 1.5 \, $a\~nos$^{-1} $ con un desv\'io est\'andar porcentual del $ 10\% $ y
$ \mu = N(1.2 \, $a\~nos$^{-1}, 0.05) $ es una funci\'on de distribuci\'on de probabilidad normal de media [ESTO ESTA BIEN?]
$ 1.2 \, $a\~nos$^{-1} $ con un desv\'io lineal porcentual de $ 5\% $ se estima
la funci\'on densidad de probabilidad del \emph{Per\'iodo}. El resultado obtenido al correr
[INSERTE NUMERO DE SIMULACIONES AQUI] se puede ver en la Figura \ref{fig:mc}.

A partir del resultado obtenido se estima la probabilidad de que el per\'odo del ciclo
poblacional sea menor o igual que 1, 5, 10 y 15 a\~nos obteniendose como resultado
[INSERTE RESULTADOS AQUI] ... respectivamente.
\begin{figure}
\centering
\label{fig:mc}
%\epsfig{file=mc.eps, scale=0.4}
\caption{Funci\'on densidad de probabilidad del periodo. Cantidad de ocurrencias versus per\'iodo.}
\end{figure}

\section{Resultados y conclusiones}
\label{sec:resultados}
Inserte resultados y conclusiones aqu\'i.
\section*{REFERENCIAS}
\textit{Diaz, "Simulaci\'on de Montecarlo", 2011} \\
\textit{Mathworks, "http://www.mathworks.com/help/techdoc/ref/ode45.html", 2011}
\end{document}


